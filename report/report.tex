\documentclass{bmcart}

%%%%%%%%%%%%%%%%%%%%%%%%%%%%%%%%%%%%%%%%%%%%%%
%%                                          %%
%% CARGA DE PAQUETES DE LATEX               %%
%%                                          %%
%%%%%%%%%%%%%%%%%%%%%%%%%%%%%%%%%%%%%%%%%%%%%%

%%% Load packages
\usepackage{amsthm,amsmath}
\usepackage{graphicx}
%\RequirePackage[numbers]{natbib}
%\RequirePackage{hyperref}
\usepackage[utf8]{inputenc} %unicode support
%\usepackage[applemac]{inputenc} %applemac support if unicode package fails
%\usepackage[latin1]{inputenc} %UNIX support if unicode package fails


%%%%%%%%%%%%%%%%%%%%%%%%%%%%%%%%%%%%%%%%%%%%%%
%%                                          %%
%% COMIENZO DEL DOCUMENTO                   %%
%%                                          %%
%%%%%%%%%%%%%%%%%%%%%%%%%%%%%%%%%%%%%%%%%%%%%%

\begin{document}

	\begin{frontmatter}
	
		\begin{fmbox}
			\dochead{Research}
			
			%%%%%%%%%%%%%%%%%%%%%%%%%%%%%%%%%%%%%%%%%%%%%%
			%% INTRODUCIR TITULO PROYECTO               %%
			%%%%%%%%%%%%%%%%%%%%%%%%%%%%%%%%%%%%%%%%%%%%%%
			
			\title{A sample article title}
			
			%%%%%%%%%%%%%%%%%%%%%%%%%%%%%%%%%%%%%%%%%%%%%%
			%% AUTORES. METER UNA ENTRADA AUTHOR        %%
			%% POR PERSONA                              %%
			%%%%%%%%%%%%%%%%%%%%%%%%%%%%%%%%%%%%%%%%%%%%%%
			
			\author[
			  addressref={aff1},                   % ESTA LINEA SE COPIA IGUAL PARA CADA AUTOR
			  corref={aff1},                       % ESTA LINEA SOLO DEBE TENERLA EL COORDINADOR DEL GRUPO
			  email={jane.e.doe@cambridge.co.uk}   % VUESTRO CORREO ACTIVO
			]{\inits{J.E.}\fnm{Jane E.} \snm{Doe}} % inits: INICIALES DE AUTOR, fnm: NOMBRE DE AUTOR, snm: APELLIDOS DE AUTOR
			\author[
			  addressref={aff1},
			  email={john.RS.Smith@cambridge.co.uk}
			]{\inits{J.R.S.}\fnm{John R.S.} \snm{Smith}}
			
			%%%%%%%%%%%%%%%%%%%%%%%%%%%%%%%%%%%%%%%%%%%%%%
			%% AFILIACION. NO TOCAR                     %%
			%%%%%%%%%%%%%%%%%%%%%%%%%%%%%%%%%%%%%%%%%%%%%%
			
			\address[id=aff1]{%                           % unique id
			  \orgdiv{ETSI Informática},             % department, if any
			  \orgname{Universidad de Málaga},          % university, etc
			  \city{Málaga},                              % city
			  \cny{España}                                    % country
			}
		
		\end{fmbox}% comment this for two column layout
		
		\begin{abstractbox}
		
			\begin{abstract} % abstract
			
			%%%%%%%%%%%%%%%%%%%%%%%%%%%%%%%%%%%%%%%%%%%%%%%
			%% RESUMEN BREVE DE NO MAS DE 100 PALABRAS   %%
			%%%%%%%%%%%%%%%%%%%%%%%%%%%%%%%%%%%%%%%%%%%%%%%	
			
			\end{abstract}
			
			%%%%%%%%%%%%%%%%%%%%%%%%%%%%%%%%%%%%%%%%%%%%%%
			%% PALABRAS CLAVE DEL PROYECTO              %%
			%%%%%%%%%%%%%%%%%%%%%%%%%%%%%%%%%%%%%%%%%%%%%%
			
			\begin{keyword}
			\kwd{sample}
			\kwd{article}
			\kwd{author}
			\end{keyword}
		
		
		\end{abstractbox}
	
	\end{frontmatter}
	
	%%%%%%%%%%%%%%%%%%%%%%%%%%%%%%%%%%%%%%%%%%%%%%%%%%%%%%%%%%%%%%%%%%%%%%%%%%%%%%%%%%%%%%%%
	%% EJEMPLO DE LATEX %%                                                                %%
	%% BORRAR ANTES DE ENTREGAR!!!!!!!!!!!!!!!!!!!!!!!!!!!!!!!!!!!!!!!!!!!!!              %%
	%%%%%%%%%%%%%%%%%%%%%%%%%%%%%%%%%%%%%%%%%%%%%%%%%%%%%%%%%%%%%%%%%%%%%%%%%%%%%%%%%%%%%%%%
	
	\section*{Content}
		Text and results for this section, as per the individual journal's instructions for authors. Here, we reference the figure \ref{fig:cost_genome} and figure \ref{fig:cost_megabase} but also the table \ref{tab:ejemplo}.
	
	\section*{Section title}
		Text for this section\ldots

		In this section we examine the growth rate of the mean of $Z_0$, $Z_1$ and $Z_2$. In
		addition, we examine a common modeling assumption and note the
		importance of considering the tails of the extinction time $T_x$ in
		studies of escape dynamics.
		We will first consider the expected resistant population at $vT_x$ for
		some $v>0$, (and temporarily assume $\alpha=0$)
		%
		\[
		E \bigl[Z_1(vT_x) \bigr]=
		\int_0^{v\wedge
			1}Z_0(uT_x)
		\exp (\lambda_1)\,du .
		\]
		%
		If we assume that sensitive cells follow a deterministic decay
		$Z_0(t)=xe^{\lambda_0 t}$ and approximate their extinction time as
		$T_x\approx-\frac{1}{\lambda_0}\log x$, then we can heuristically
		estimate the expected value as
		%
		\begin{equation}\label{eqexpmuts}
			\begin{aligned}[b]
				&      E\bigl[Z_1(vT_x)\bigr]\\
				&\quad      = \frac{\mu}{r}\log x
				\int_0^{v\wedge1}x^{1-u}x^{({\lambda_1}/{r})(v-u)}\,du .
			\end{aligned}
		\end{equation}
		%
		%%%%%%%%%%%%%%%%%%%%%%%%%%%%%%%%%%%%%%%%%%%%%%%%%%%%%%%%%%%%%%%%%%%%%%
		%% USAR \cite{...} PARA INCLUIR REFERENCIAS BIBLIOGRAFICAS          %%
		%%  \cite{koon}  Para una sola                                      %%
		%%  \cite{oreg,khar,zvai,xjon,schn,pond} Para una lista             %%
		%%%%%%%%%%%%%%%%%%%%%%%%%%%%%%%%%%%%%%%%%%%%%%%%%%%%%%%%%%%%%%%%%%%%%%
		Thus we observe that this expected value is finite for all $v>0$ (also see \cite{koon,xjon,marg,schn,koha,issnic}).
		
		
		%%%%%%%%%%%%%%%%%%%%%%%%%%%%%%%%%%%%%%%%%%%%%%%%%%%%%%%%%%%%%%%%%%%%%%%%%%%%%%%%%%%%%%%%%%%
		%% FIGURAS                                                                               %%
		%% includegraphics: inserta la imagen                                                    %%
		%% caption: descripcion de la figura                                                     %%
		%% label: etiqueta para hacer referencia a la figura en el texto con la instrucción ref  %%
		%%%%%%%%%%%%%%%%%%%%%%%%%%%%%%%%%%%%%%%%%%%%%%%%%%%%%%%%%%%%%%%%%%%%%%%%%%%%%%%%%%%%%%%%%%%	
		
		\begin{figure}[h!]
			\includegraphics[width=0.9\textwidth]{figures/Sequencing_Cost_per_Genome_May2020.jpg}
			\caption{Sample figure title}
			\label{fig:cost_genome}
		\end{figure}
		
		\begin{figure}[h!]
			\includegraphics[width=0.9\textwidth]{figures/Sequencing_Cost_per_Megabase_May2020.jpg}
			\caption{Sample figure title}
			\label{fig:cost_megabase}
		\end{figure}
		
		%%%%%%%%%%%%%%%%%%%%%%%%%%%%%%%%%%%%%%%%%%%%%%%%%%%%%%%%%%%%%%%%%%%%%%%%%%%%%%%%%%%%%%%%%%
		%% TABLAS                                                                               %%
		%% caption: Descripción tabla                                                           %%
		%% \begin{tabular}{letras}: Indica numero de columnas.                                  %%
		%%    Una letra por columna, la letra indica la alineación de la columna:               %%
		%%    c center, l left, r right                                                         %%
		%% hline: Representa una linea como entre filas                                         %%
		%% \\: fin de fila                                                                      %%
		%% &: delimitador de celda                                                              %%
		%% label: etiqueta para hacer referencia a la tabla en el texto con la instrucción ref  %%
		%%%%%%%%%%%%%%%%%%%%%%%%%%%%%%%%%%%%%%%%%%%%%%%%%%%%%%%%%%%%%%%%%%%%%%%%%%%%%%%%%%%%%%%%%%
		
		\begin{table}[h!]
			\caption{Sample table title. This is where the description of the table should go}
			\begin{tabular}{cccc}
				\hline
				& B1  &B2   & B3\\ 
				\hline
				A1 & 0.1 & 0.2 & 0.3\\
				A2 & ... & ..  & .\\
				A3 & ..  & .   & .\\ 
				\hline
				\label{tab:ejemplo}
			\end{tabular}
		\end{table}
				
		\subsection*{Sub-heading for section}
			Text for this sub-heading\ldots
	
			\subsubsection*{Sub-sub heading for section}
				Text for this sub-sub-heading\ldots
					
					\paragraph*{Sub-sub-sub heading for section}
						Text for this sub-sub-sub-heading\ldots
	
	%%%%%%%%%%%%%%%%%%%%%%%%%%%%%%%%%
	%% FIN DE EJEMPLO !!!!!!!!!!!! %%
	%%%%%%%%%%%%%%%%%%%%%%%%%%%%%%%%%
	
	%%%%%%%%%%%%%%%%%%%%%%%%%%%%%%%%%
	%% COMIENZO DEL DOCUMENTO REAL %%
	%%%%%%%%%%%%%%%%%%%%%%%%%%%%%%%%%
	
	\input{tex_files/introduction.tex}
	\section{Materiales y métodos}

	\input{tex_files/resultados.tex}
	\input{tex_files/discusion.tex}
	\input{tex_files/conclusiones.tex}
	
	
	%%%%%%%%%%%%%%%%%%%%%%%%%%%%%%%%%%%%%%%%%%%%%%
	%% OTRA INFORMACIÓN                         %%
	%%%%%%%%%%%%%%%%%%%%%%%%%%%%%%%%%%%%%%%%%%%%%%
	
	\begin{backmatter}
	
		\section*{Abreviaciones}%% if any
			Indicar lista de abreviaciones mostrando cada acrónimo a que corresponde
		
		\section*{Disponibilidad de datos y materiales}%% if any
			Debéis indicar aquí un enlace a vuestro repositorio de github.
		
		\section*{Contribución de los autores}
			Usando las iniciales que habéis definido al comienzo del documento, debeis indicar la contribución al proyecto en el estilo:
			J.E : Encargado del análisis de coexpresión con R, escritura de resultados; J.R.S : modelado de red con python y automatizado del código, escritura de métodos; ...
			OJO: que sea realista con los registros que hay en vuestros repositorios de github. 
		
		
		%%%%%%%%%%%%%%%%%%%%%%%%%%%%%%%%%%%%%%%%%%%%%%%%%%%%%%%%%%%%%%%%%%%%%%%%%%%%%%%%%%%%%%%%
		%% BIBLIOGRAFIA: no teneis que tocar nada, solo sustituir el archivo bibliography.bib %%
		%% por el que hayais generado vosotros                                                %%
		%%%%%%%%%%%%%%%%%%%%%%%%%%%%%%%%%%%%%%%%%%%%%%%%%%%%%%%%%%%%%%%%%%%%%%%%%%%%%%%%%%%%%%%%
		
		\bibliographystyle{bmc-mathphys} % Style BST file (bmc-mathphys, vancouver, spbasic).
		\bibliography{bibliography}      % Bibliography file (usually '*.bib' )
	
	\end{backmatter}
\end{document}
